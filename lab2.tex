\documentclass[12pt]{amsart}

\title{First Order Autonomous Differential Equations}

\begin{document}
\maketitle

\section{Introduction}
A first order, autonomous differential equation is of the form:
$$
y'=f(y)
$$
\begin{enumerate}
\item A direction field is a plot of local slopes where the axes
are $t$ and $y$.  A {\it phase plot} is a plot of $y'$ versus $y$.
\item An {\it equilibrium solution} is a constant solution,
$y(t)=c$ so that $f(c)=0$.  These solutions are also called {\it
fixed points} or {\it critical points}.
\item An equilibrium solution is said to be an {\it attracting}
solution if there is an interval of initial values about $y=c$ so
that all solutions tend to $c$ as $t\rightarrow \infty$.
\item An equilibrium solution is said to be a {\it repelling}
solution if there is an interval about $y=c$ so that all solutions
tend away from $y=c$.
\end{enumerate}

\section{Goal of the Lab}
The goal of this lab is to explore first order autonomous
differential equations, learn some vocabulary associated with
them, and to do graphical analysis of the solutions.

\section{Lab Questions}
\begin{enumerate}
\item From your last lab, you discussed the direction field for
DEs of the form $y'=f(y)$.  Use your conclusions to discuss the
following observation:

\begin{quote}
``No solution to $y'=f(y)$ can oscillate.  All solutions will
either be constant, monotone increasing, or monotone decreasing.''
\end{quote}

\item The relationship of the phase plot to the direction field:
Suppose we are given the differential equation:
$$
y'=f(y)=\frac{1}{10}y(3-y)(y+1)^2
$$
\begin{enumerate}
\item Graph $f(y)$ versus $y$ and compare this to the direction
field.  Is it possible to predict the long term behavior of a
particular solution by its position in the phase plot?
\item What are the equilibrium solutions?  Classify each as {\it
attracting, repelling,} or {\it neither}.  For each equilibrium,
was it possible to predict this based only on the phase plot?
Hint:  Consider $\frac{df}{dy}$ at each equilibrium.
\item Comment on the following observation for autonomous, first
order equations:
\begin{quote}
``The behavior of all solutions to $y'=f(y)$ is organized around
the equilibrium solutions''
\end{quote}
\end{enumerate}
\item Consider the family of differential equations of the form:
$$
y'=ky-y^3
$$
where $k$ is a constant.
\begin{enumerate}
\item Solve for the equilibrium solutions in terms of $k$ (you can
do this by hand).
\item Perform a phase plot for different values of $k$, and state
in words the effect that $k$ has on the graph of $ky-y^3$.
Consider the following situations:
\begin{itemize}
\item If $k<0$, how many equilibria are there, and what type are
they?
\item If $k=0$, answer the same question.
\item If $k>0$, answer the same question.
\end{itemize}
\end{enumerate}

{\it Side Remark:  The phenomena you are talking about is called a
{\bf bifurcation}- that is, changing a parameter results in the
creation/destruction of equilibrium solutions!  In fact, this
particular situation is called a {\bf pitchfork bifurcation}.}
\item Some general questions:
\begin{enumerate}
\item Is it possible to have two attracting equilibria with no
repelling equilibrium in between?  To answer this, consider the
phase plot of $f(y)$ versus $y$.
\item Construct your own autonomous, first order differential
equation modeling population, that would have the correspond to
the following behavior:

``If the population falls below 2, the population dies off.  If
the population is above 2 and less than 20, the population will
tend to 20.  If the population is greater than 20, there is not
enough food so the population will tend back towards 20.''

Verify the behavior by providing a phase plot and direction field.


\end{enumerate}
\end{enumerate}

\end{document}
